% In this file you should put the actual content of the blueprint.
% It will be used both by the web and the print version.
% It should *not* include the \begin{document}
%
% If you want to split the blueprint content into several files then
% the current file can be a simple sequence of \input. Otherwise It
% can start with a \section or \chapter for instance.

\chapter{Definitions}
\label{chap:def}

\begin{definition}
    \label{def:class}
    \lean{MvPolynomial.cls}
    \leanok
    The class of a multivariate polynomial $p$ is the largest variable index appearing in $p$.
\end{definition}

\begin{definition}
    \label{def:degree}
    \lean{MvPolynomial.degree}
    \leanok
    \uses{def:class}
    The degree of $p$ with respect to its class.
\end{definition}

\begin{definition}
    \label{def:MvPolynomial.rank}
    \lean{MvPolynomial.rank}
    \leanok
    \uses{def:class, def:degree}
    The rank of a polynomial $p$ is the pair $(cls(p), deg(p))$ ordered lexicographically.
\end{definition}

\begin{definition}
    \label{def:reducedTo}
    \lean{MvPolynomial.reducedTo}
    \leanok
    \uses{def:degree, def:class}
    $q$ is reduced with respect to $p$ if the degree of $q$ in the main variable of $p$ is strictly less than the main degree of $p$.
\end{definition}

\begin{definition}
    \label{def:reducedToSet}
    \lean{MvPolynomial.reducedToSet}
    \leanok
    \uses{def:reducedTo}
    $q$ is reduced with respect to a polynomial set $PS$ if it is reduced with respect to all elements of $PS$.
\end{definition}

\begin{definition}
    \label{def:initialOf}
    \lean{MvPolynomial.initialOf}
    \leanok
    \uses{def:degree}
    The initial of a polynomial $p$ with respect to a variable $i$.
    It is the coefficient of the highest power of $x_i$ appearing in $p$.
\end{definition}

\begin{definition}
    \label{def:initial}
    \lean{MvPolynomial.initial}
    \leanok
    \uses{def:class, def:initialOf}
    The initial of a polynomial $p$ is the initial with respect to its class.
\end{definition}

\begin{definition}
    \label{def:initialProd}
    \lean{MvPolynomial.initialProd}
    \leanok
    \uses{def:initial}
    The product of initials of a set of polynomials.
\end{definition}

\begin{definition}
    \label{def:TriangulatedSet}
    \lean{TriangulatedSet}
    \leanok
    \uses{def:class}
    A Triangulated Set is a finite ordered sequence of non-zero polynomials with strictly increasing classes.
\end{definition}

\begin{definition}
    \label{def:TriangulatedSet.rank}
    \lean{TriangulatedSet.rank}
    \leanok
    \uses{def:TriangulatedSet, def:MvPolynomial.rank}
    The rank of a Triangulated Set is a lexicographic sequence of ranks of its polynomials.
    More intuitively, $S < T$ if one of the following two occurs:
    \begin{itemize}
        \item There exists some $k < S.length$ such that $S_0 \sim T_0, S_1 \sim T_1, ..., S_{k-1} \sim T_{k-1}$, while $S_k < T_k$.
        \item $S.length > T.length$ and $\forall i < T.length, S_i \sim T_i$.
    \end{itemize}
\end{definition}

\begin{definition}
    \label{def:isRemainder}
    \lean{MvPolynomial.isRemainder}
    \leanok
    \uses{def:reducedTo, def:initial}
    A remainder $r$ of $g$ by $f$ is a polynomial which is reduced with respect to $f$ and
    satisfies $init(f)^s \cdot g = q \cdot f + r$ for some $s \in \mathbb{N}$ and $q \in R[X_{\sigma}]$.
\end{definition}

\begin{definition}
    \label{def:isSetRemainder}
    \lean{MvPolynomial.isSetRemainder}
    \leanok
    \uses{def:reducedTo, def:initial}
    A remainder $r$ of $g$ by a set $S$ is a polynomial which is reduced with respect to $S$ and
    satisfies $(\prod S_i^{e_i}) \cdot g = \sum q_i \cdot S_i + r$ for some $\{e_i\}$ and $\{q_i\}$.
\end{definition}

\begin{definition}
    \label{def:pseudoOf}
    \lean{MvPolynomial.pseudoOf}
    \leanok
    \uses{def:initialOf}
    Pseudo-division of $g$ by $f$ with respect to the variable $x_i$.
    
    Returns a triple containing the exponent, the quotient and the remainder.
\end{definition}

\begin{definition}
    \label{def:pseudo}
    \lean{MvPolynomial.pseudo}
    \leanok
    \uses{def:class, def:pesudoOf}
    Pseudo-division of $g$ by $f$.

    If $f$ is constant, it performs standard division.
    If $f$ is non-constant, it performs pseudo-division with respect to $cls(f)$.

    Returns a triple containing the exponent, the quotient and the remainder.
\end{definition}

\begin{definition}
    \label{def:setPseudo}
    \lean{MvPolynomial.setPseudo}
    \leanok
    \uses{def:TriangulatedSet, def:initial, def:pesudo}
    Pseudo-divides $g$ successively by elements of $S$.
    Typically, this involves dividing by $S_{l-1}$, then $S_{l-2}$, ..., down to $S_0$.

    Returns a triple containing the exponents, the quotients and the remainder.
\end{definition}

\begin{definition}
    \label{def:isAscendingSet}
    \lean{TriangulatedSet.isAscendingSet}
    \leanok
    \uses{def:TriangulatedSet, def:reducedTo}
    A Triangulated Set is an Ascending Set if every element is reduced with respect to its predecessors.
    Here "reduced" is an abstract predicate.
\end{definition}

\begin{definition}
    \label{def:isStandardAscendingSet}
    \lean{StandardAscendingSet.isAscendingSet}
    \leanok
    \uses{def:reducedTo, def:TriangulatedSet}
    A Triangulated Set is a Standard Ascending Set if every element is reduced with respect to its predecessors.
\end{definition}

\begin{definition}
    \label{def:isWeakAscendingSet}
    \lean{WeakAscendingSet.isAscendingSet}
    \leanok
    \uses{def:initial, def:reducedTo, def:TriangulatedSet}
    A Triangulated Set is a Weak Ascending Set if the initial of every element is reduced with respect to its predecessors.
\end{definition}

\begin{definition}
    \label{def:isMinimal}
    \lean{isMinimal}
    \leanok
    \uses{def:TriangulatedSet.rank}
    An ascending set $AS$ is minimal in a polynomial set $PS$ if $AS \subseteq PS$ and $AS \le AS'$ for any other ascending set $AS' \subseteq PS$.
    It is also called a "Basic Set" of $PS$.
\end{definition}

\begin{definition}
    \label{def:HasBasicSet}
    \lean{HasBasicSet}
    \leanok
    \uses{def:TriangulatedSet, def:isAscendingSet, def:reducedToSet, def:TriangulatedSet.rank}
    The interface for algorithms computing Basic Sets.
    Any instance of this class provides a "basicSet" function that computes a minimal ascending set
    contained in a given list of polynomials.
\end{definition}

\begin{definition}
    \label{def:StandardAscendingSet.basicSet}
    \lean{StandardAscendingSet.basicSet}
    \leanok
    \uses{def:TriangulatedSet, def:reducedToSet, def:TriangulatedSet.rank}
    Computes the Standard Basic Set of a list of polynomials.

    The algorithm works by:
    \begin{enumerate}
        \item Sort the list and set $BS = \emptyset$.
        \item Pick the first (minimal) element $B$ in the list.
        \item Append $B$ to the tail of the current basic set $BS$.
        \item Filter the remaining list to keep only elements reduced w.r.t. the new $BS$ and go to step 2.
    \end{enumerate}
\end{definition}

\begin{definition}
    \label{def:WeakAscendingSet.basicSet}
    \lean{WeakAscendingSet.basicSet}
    \leanok
    \uses{def:class, def:initial, def:TriangulatedSet, def:reducedToSet, def:TriangulatedSet.rank}
    Computes the Weak Basic Set of a list of polynomials.

    Difference from Standard:
    The filter condition includes $cls(p) > cls(B)$.

    This is because $init(p) reduced w.r.t. B$ does NOT imply $cls(p) > cls(B)$ (unlike strong reduction).
    We must enforce the triangular structure explicitly.
\end{definition}

\begin{definition}
    \label{def:isCharacteristicSet}
    \lean{TriangulatedSet.isCharacteristicSet}
    \leanok
    \uses{def:TriangulatedSet, def:isSetRemainder}
    $CS$ is a characteristic set of $PS$ if every polynomial in $PS$ reduce to 0 modulo $CS$, and $Zero(PS) \subseteq Zero(CS)$.
\end{definition}

\begin{definition}
    \label{def:characteristicSet}
    \lean{MvPolynomial.List.characteristicSet}
    \leanok
    \uses{def:TriangulatedSet, def:HasBasicSet, def:setPseudo, inst:TriangulatedSet.wellFoundedLT}
    Computes the Characteristic Set of a polynomial list $l$.

    Algorithm:
    \begin{enumerate}
        \item Set $l_0 = l$.
        \item Compute $BS = BasicSet(l)$.
        \item Compute remainders $RS$ of $l \setminus BS$ with respect to $BS$.
        \item If $RS = \emptyset$, $BS$ is the characteristic set.
        \item If not, set $l = l₀ ++ RS ++ BS$ and go to step 2.
    \end{enumerate}
    Termination is guaranteed by the well-ordering of ranks.
\end{definition}

\begin{definition}
    \label{def:zeroDecomposition}
    \lean{MvPolynomial.List.zeroDecomposition}
    \leanok
    \uses{def:characteristicSet}
    Decomposes the zero set of a polynomial list into a union of zero sets of triangular sets.
    The algorithm recursively computes the characteristic set $CS$
    and adds branches for the initials of $CS$.
\end{definition}

\chapter{Theorems}
\label{chap:thm}

\begin{theorem}
    \label{thm:initialOf_decomposition}
    \lean{MvPolynomial.initialOf_decomposition}
    \leanok
    \uses{def:initialOf}
    $p = init_i(p) x_i ^ d + q$, where $deg_i(q) < d = deg_i(p)$.
\end{theorem}

\begin{theorem}
    \label{inst:MvPolynomial.wellFoundedLT}
    \lean{MvPolynomial.instWellFoundedLT}
    \leanok
    \uses{def:MvPolynomial.rank}
    The polynomial ring is well-founded under the rank ordering (given well-founded variables).
\end{theorem}

\begin{theorem}
    \label{inst:TriangulatedSet.wellFoundedLT}
    \lean{TriangulatedSet.instWellFoundedLT}
    \leanok
    \uses{def:TriangulatedSet, def:TriangulatedSet.rank}
    The set of Triangulated Sets is well-founded under the lexicographic rank ordering. This guarantees the termination of the Characteristic Set Algorithm.
\end{theorem}

\begin{theorem}
    \label{thm:pseudo_remainder_isRemainder}
    \lean{MvPolynomial.pseudo_remainder_isRemainder}
    \leanok
    \uses{def:isRemainder, def:pseudo}
    The remainder $r$ of pseudo-division of $g$ by $f$ is reduced with respect to $f$ and
    satisfies $init(f)^s \cdot g = q \cdot f + r$ for some $s \in \mathbb{N}$ and $q \in R[X_{\sigma}]$.
\end{theorem}

\begin{theorem}
    \label{thm:setPseudo_remainder_isSetRemainder}
    \lean{MvPolynomial.setPseudo_remainder_isSetRemainder}
    \leanok
    \uses{def:isSetRemainder, def:setPseudo}
    A remainder $r$ of pseudo-division of $g$ by a set $S$ is reduced with respect to $S$ and
    satisfies $(\prod S_i^{e_i}) \cdot g = \sum q_i \cdot S_i + r$ for some $\{e_i\}$ and $\{q_i\}$.
\end{theorem}

\begin{theorem}
    \label{inst:StandardAscendingSet.instHasBasicSet}
    \lean{StandardAscendingSet.instHasBasicSet}
    \leanok
    \uses{def:isStandardAscendingSet, def:isMinimal, def:StandardAscendingSet.basicSet}
    The algorithm computes a minimal standard ascending set contained in the input list.
\end{theorem}

\begin{theorem}
    \label{inst:WeakAscendingSet.instHasBasicSet}
    \lean{WeakAscendingSet.instHasBasicSet}
    \leanok
    \uses{def:isWeakAscendingSet, def:isMinimal, def:WeakAscendingSet.basicSet}
    The algorithm computes a minimal weak ascending set contained in the input list.
\end{theorem}

\begin{theorem}
    \label{thm:vanishingSet_diff_initialProd_subset}
    \lean{CharacteristicSet.vanishingSet_diff_initialProd_subset}
    \leanok
    \uses{def:TriangulatedSet, def:isCharacteristicSet, def:initialProd}
    Well-ordering principle (1):
    $Zero(CS/IP) \subseteq Zero(PS)$, where $IP$ is the initial-product of $CS$.
\end{theorem}

\begin{theorem}
    \label{thm:vanishingSet_diff_initialProd_eq}
    \lean{CharacteristicSet.vanishingSet_diff_initialProd_eq}
    \leanok
    \uses{def:TriangulatedSet, def:isCharacteristicSet, def:initialProd}
    Well-ordering principle (2):
    $Zero(CS/IP) = Zero(PS/IP)$, where $IP$ is the initial-product of $CS$.
\end{theorem}

\begin{theorem}
    \label{thm:vanishingSet_decomposition}
    \lean{CharacteristicSet.vanishingSet_decomposition}
    \leanok
    \uses{def:TriangulatedSet, def:isCharacteristicSet ,def:initial}
    Well-ordering principle (3):
    $Zero(PS) = (Zero(CS/IP)) \cup \bigcup_{p \in CS} (Zero(PS) \cap Zero(init(p)))$.
\end{theorem}

\begin{theorem}
    \label{thm:characteristicSet_isCharacteristicSet}
    \lean{MvPolynomial.List.characteristicSet_isCharacteristicSet}
    \leanok
    \uses{def:characteristicSet, def:isCharacteristicSet}
    The algorithm computes a valid characteristic set for the input list.
\end{theorem}

\begin{theorem}
    \label{thm:characteristicSet_isAscendingSet}
    \lean{MvPolynomial.List.characteristicSet_isAscendingSet}
    \leanok
    \uses{def:characteristicSet, def:isAscendingSet}
    The computed characteristicSet is an ascending set.
\end{theorem}

\begin{theorem}
    \label{thm:characteristicSet_le_basicSet}
    \lean{MvPolynomial.List.characteristicSet_le_basicSet}
    \leanok
    \uses{def:characteristicSet, def:TriangulatedSet.rank, def:HasBasicSet}
    The rank of computed characteristicSet $\le$ the rank of the input list.
\end{theorem}

\begin{theorem}
    \label{thm:vanishingSet_eq_zeroDecomposition_union}
    \lean{MvPolynomial.List.vanishingSet_eq_zeroDecomposition_union}
    \leanok
    \uses{def:isCharacteristicSet, def:zeroDecomposition, thm:characteristicSet_isCharacteristicSet, thm:vanishingSet_decomposition, inst:TriangulatedSet.wellFoundedLT}
    \textbf{Zero Decomposition Theorem}:
    The zero set of a polynomial system $PS$ is the union of the zero sets of the triangular systems computed by the algorithm:
    $$ Zero(PS) = \bigcup_{CS \in \mathcal{ZD}} Zero(CS/IP(CS)) $$
\end{theorem}